%%%%%%%%%%%%%%%%%%%%%%%%%%%%%%%%%%%%%%%%%%%%%%%%%%%%%%%%%%%%%%%%%%%%%%%%%
%% Trim Size: 9.75in x 6.5in
%% Text Area: 8in (include Runningheads) x 5in
%% ws-jmmb.tex   :   17-5-2004
%% TeX file to use with ws-jmmb.cls written in Latex2E. 
%% The content, structure, format and layout of this style file is the 
%% property of World Scientific Publishing Co. Pte. Ltd. 
%% Copyright 1995, 2002 by World Scientific Publishing Co. 
%% All rights are reserved.
%%%%%%%%%%%%%%%%%%%%%%%%%%%%%%%%%%%%%%%%%%%%%%%%%%%%%%%%%%%%%%%%%%%%%%%%%
%%

\documentclass{ws-jmmb}

\usepackage{epsfig}

% \tikzexternalize[prefix=tikz/,optimize command away=\includepdf]

\begin{document}

\markboth{Authors' Names}{Instructions for 
Typing Manuscripts (Paper's Title)}

%%%%%%%%%%%%%%%%%%%%% Publisher's Area please ignore %%%%%%%%%%%%%%%
%
\catchline{}{}{}{}{}
%
%%%%%%%%%%%%%%%%%%%%%%%%%%%%%%%%%%%%%%%%%%%%%%%%%%%%%%%%%%%%%%%%%%%%

\title{INSTRUCTIONS FOR TYPESETTING MANUSCRIPTS\\
USING COMPUTER SOFTWARE\footnote{For the title, try not to use more
than 3 lines.  Typeset the title in 10 pt roman, uppercase and
boldface.}  }

\author{FIRST AUTHOR\footnote{Typeset names in 
8 pt roman, uppercase. Use the footnote to indicate the
present or permanent address of the author.}}

\address{University Department, University Name, Address\\
City, State ZIP/Zone,Country\,\footnote{State completely without
abbreviations, the affiliation and mailing address, including
country. Typeset in 8 pt italic.}\\
\email{author\_id@domain\_name\footnote{Typeset author e-mail address 
in single line.}}
\http{$<$webaddress$>$} }

\author{SECOND AUTHOR}

\address{Group, Laboratory, Address\\
City, State ZIP/Zone, Country\\
author\_id@domain\_name
}

\maketitle

\begin{history}
\received{(Day Month Year)}
\accepted{(Day Month Year)}
\end{history}

\begin{abstract}
The abstract should summarize the context, content and conclusions of
the paper in less than 200 words. It should not contain any reference
citations or displayed equations. Typeset the abstract in 8 pt roman
with baselineskip of 10 pt, making an indentation of 1.5 pica on the
left and right margins.
\end{abstract}

\keywords{Keyword1; keyword2; keyword3.}

\ccode{1991 Mathematics Subject Classification: 22E46, 53C35, 57S20}

\section{General Appearance}	%) A SECTION HEADING

Contributions to the {\it Journal of Mechanics in Medicine and
Biology} will mostly be processed by using the \hbox{authors'} source
files. These should be submitted with the manuscripts, and resubmitted
in the final form if a paper requires revision before being accepted
for publication.

\section{The Main Text}

Authors are encouraged to have their contribution checked for grammar.
American spelling should be used. Abbreviations are allowed but should
be spelled out in full when first used. Integers ten and below are to be
spelled out. Italicize foreign language phrases (e.g.~Latin, French).

The text is to be typeset in 10 pt roman, single spaced with
baselineskip of 13~pt. Text area including running head is 8 inches in
length and 5 inches in width. Final pagination and insertion of
running titles will be done by the publisher.

\section{Major Headings}

Major headings should be typeset in boldface with the first letter of
important words capitalized.

\subsection{Sub-headings}

Sub-headings should be typeset in boldface italic and capitalize
the first letter of the first word only. Section number to be in
boldface roman.

\subsubsection{Sub-subheadings}

Typeset sub-subheadings in medium face italic and capitalize the
first letter of the first word only. Section numbers to be in
roman.

\subsection{Numbering and spacing}

Sections, sub-sections and sub-subsections are numbered in
Arabic.  Use double spacing before all section headings, and
single spacing after section headings. Flush left all paragraphs
that follow after section headings.

\subsection{Lists of items}

List may be presented with each item marked by bullets and numbers.

\subsection*{Bulleted items}

\begin{itemlist}
 \item item one,
 \item item two,
 \item item three.
\end{itemlist}

\subsection*{Numbered items}

\begin{arabiclist}
 \item item one,
 \item item two,
 \item item three,
\end{arabiclist}

The order of subdivisions of items in bullet and numbered lists
may be presented as follows:

\subsection*{Bulleted items}

\begin{itemize}
\item First item in the first level
\item Second item in the first level
\begin{itemize}
\item First item in the second level 
\item Second item in the second level
\begin{itemize}
\item First item in the third level 
\item Second item in the third level
\end{itemize}
\item Third item in the second level
\item fourth item in the second level
\end{itemize}
\item third item in the first level
\item fourth item in the first level
\end{itemize}

\subsection*{Numbered items}

\begin{arabiclist}
\item First item in the first level
\item Second item in the first level
\begin{alphlist}[(a)]
\item First item in the second level 
\item Second item in the second level
\begin{romanlist}[iii.]
\item First item in the third level 
\item Second item in the third level
\item Third item in the third level
\end{romanlist}
\item Third item in the second level
\item fourth item in the second level
\end{alphlist}
\item third item in the first level
\item fourth item in the first level
\end{arabiclist}

\section{Equations}

Displayed equations should be numbered consecutively,
with the number set flush right and enclosed in parentheses. The
equation numbers should be consecutive within the contribution.
\begin{equation}
\mu(n, t) = {\sum^\infty_{i=1} 1(d_i < t, N(d_i) = n) \over
\int^t_{\sigma=0} 1(N(\sigma) = n)d\sigma}\,.
\label{eq:jaa}
\end{equation}

Equations should be referred to in abbreviated form,
e.g.~``Eq.~(\ref{eq:jaa})'' or ``(2)''. In multiple-line
equations, the number should be given on the last line.

Displayed equations are to be centered on the page width.
Standard English letters like x are to appear as $x$
(italicized) in the text if they are used as mathematical
symbols. Punctuation marks are used at the end of equations as
if they appeared directly in the text.

\section{Theorem environments}

\begin{theorem}
Theorems, lemmas, definitions, etc. are set on a separate paragraph,
with extra 1 line space above and below. They are to be numbered
consecutively within the contribution.
\end{theorem}

\begin{lemma}
Theorems, lemmas, definitions, etc. are set on a separate paragraph,
with extra 1 line space above and below. They are to be numbered
consecutively within the contribution.
\end{lemma}

\begin{proof}
Proofs should end with
\end{proof}

\section{Illustrations and Photographs}

\begin{figure}[b]
\centerline{\psfig{file=jmmbf1.eps,width=5cm}}
\vspace*{8pt}
\caption{A schematic illustration of dissociative recombination. The
direct mechanism, 4m$^2_\pi$ is initiated when the
molecular ion S$_{\rm L}$ captures an electron with kinetic energy.}
\end{figure}

Figures are to be inserted in the text nearest their first reference.
Figure placements can be either top or bottom.  Original india ink
drawings of glossy prints are preferred. Please send one set of
originals with copies. If the author requires the publisher to reduce
the figures, ensure that the figures (including letterings and
numbers) are large enough to be clearly seen after reduction. If
photographs are to be used, only black and white ones are acceptable.

Figures are to be sequentially numbered in Arabic numerals. The
caption must be placed below the figure. Typeset in 8 pt roman with
baselineskip of 10~pt. Long captions are to be justified by the
``page-width''.  Use double spacing between a caption and the text
that follows immediately.

Previously published material must be accompanied by written
permission from the author and publisher.

\section{Tables}

Tables should be inserted in the text as close to the point of
reference as possible. Some space should be left above and below
the table.

Tables should be numbered sequentially in the text in Arabic
numerals. Captions are to be centralized above the tables.  Typeset
tables and captions in 8 pt roman with baselineskip of 10 pt. Long
captions are to be justified by the ``table-width''.

\begin{table}[h]
\tbl{Comparison of acoustic for frequencies for piston-cylinder problem.}
{\begin{tabular}{@{}cccc@{}}
\toprule
Piston mass & Analytical frequency & TRIA6-$S_1$ model &
\% Error \\
& (Rad/s) & (Rad/s) \\
\colrule
1.0\hphantom{00} & \hphantom{0}281.0 & \hphantom{0}280.81 & 0.07 \\
0.1\hphantom{00} & \hphantom{0}876.0 & \hphantom{0}875.74 & 0.03 \\
0.01\hphantom{0} & 2441.0 & 2441.0\hphantom{0} & 0.0\hphantom{0} \\
0.001 & 4130.0 & 4129.3\hphantom{0} & 0.16\\
\botrule
\end{tabular}}
\begin{tabnote}
Table notes
\end{tabnote}
\begin{tabfootnote}[]
$^{\rm a}$Table footnote A\\
$^{\rm b}$Table footnote B
\end{tabfootnote}
\end{table}

If tables need to extend over to a second page, the continuation of
the table should be preceded by a caption, e.g.~``{\it Table 2.}
$(${\it Continued}$)$''. Notes to tables are placed below the final
row of the table and should be flushleft.  Footnotes in tables
shouldbe indicated by superscript lowercase letters and placed beneath
the table.


\section{Running Heads}

Please provide a shortened runninghead (not more than eight words) for
the title of your paper. This will appear on the top right-hand side
of your paper.

\section{Footnotes}

Footnotes should be numbered sequentially in superscript
lowercase roman letters.\footnote{Footnotes should be
typeset in 8 pt roman at the bottom of the page.}


\section*{Acknowledgments}

This section should come before the References. Funding
information may also be included here.


\appendix

\section{Appendices}

Appendices should be used only when absolutely necessary. They should
come after the References. If there is more than one appendix, number
them alphabetically. Number displayed equations occurring in the
Appendix in this way, e.g.~(\ref{appeqn}), (A.2), etc.
\begin{equation}
\mu(n, t) = {\sum^\infty_{i=1} 1(d_i < t, N(d_i) = n) \over
\int^t_{\sigma=0} 1(N(\sigma) = n)d\sigma}\,.
\label{appeqn}
\end{equation}

\section*{References}

References are to be listed in the order cited in the text in Arabic
numerals. They can be typed in superscripts after punctuation marks,
e.g.~``$\ldots$ in the statement.\cite{joliat}'' or used directly,
e.g.~``see Ref.~5 for examples.'' Please list using the style shown in
the following examples.  For journal names, use the standard
abbreviations.  Typeset references in 9 pt roman.

\begin{thebibliography}{0}
\bibitem{beeson} Beeson MJ, {\it Foundations of Constructive Mathematics}, 
Springer, Berlin, 1985.

\bibitem{clark} Clark KL, Negations as failure, in Gallaire H, 
Winker J (eds.), {\it Logic and Data Bases}, Plenum Press, New York, 
pp.~293--306, 1973.

\bibitem{tamassia} Tamassia R, Batini C, Talamo M, An 
algorithm for automatic layout of entity relationship diagrams, 
in Davis CG, Jajodia S, Ng PA, Yeh RT (eds.), 
{\it Entity-Relationship Approach to Software Engineering, 
Proc. 3rd Int. Conf. Entity-Relationship Approach}, 
North-Holland, Amsterdam, pp.~421--439, 1983.

\bibitem{gewirtz} Gewirtz WL, Investigations in the theory of 
descriptive complexity, Ph. D. Thesis, New York University, 1974.

\bibitem{joliat} Joliat M, A simple technique for partial elimination of 
unit productions from LR({\it k}) parsers, {\it IEEE Trans Comput} 
{\bf 27}:753--764, 1976.

\bibitem{lorentz} Lorentz R, Benson DB, Deterministic and 
nondeterministic flow-chart interpretations, {\it J Comput System Sci} 
{\bf 27}:400--433, 1983.

\bibitem{db} Loren R, Li J, Benson DB, Deterministic flow-chart 
interpretations, to appear in {\it J Comput Syst Sc.}

\bibitem{priv} Loren R, private communication.

\end{thebibliography}

\end{document}

