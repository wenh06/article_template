% !TEX program = xelatex
% !BIB program = bibtex

\documentclass[conference]{IEEEtran}
\IEEEoverridecommandlockouts
% The preceding line is only needed to identify funding in the first footnote. If that is unneeded, please comment it out.
\usepackage{cite}
\usepackage{amsmath,amssymb,amsfonts}
\usepackage{algorithmic}
\usepackage{graphicx}
\usepackage{textcomp}
\usepackage{xcolor}
\usepackage{tikz}
\usetikzlibrary{shapes,arrows,decorations.pathmorphing,backgrounds,positioning,fit,petri,calc}
\usepackage[colorlinks,linkcolor=black,anchorcolor=black,citecolor=black, urlcolor=blue]{hyperref}
\usepackage{subcaption}
\usepackage{cleveref}
\captionsetup[table]{skip = 3pt}
\captionsetup[subfigure]{subrefformat=simple}
\usepackage{multirow}
\usepackage{booktabs}
\usepackage{etoolbox,siunitx}
\usepackage{array}
\newcolumntype{P}[1]{>{\centering\arraybackslash}p{#1}}

\sisetup{detect-weight,mode=text}
% for avoiding siunitx using bold extended
\renewrobustcmd{\bfseries}{\fontseries{b}\selectfont}
\renewrobustcmd{\boldmath}{}
% abbreviation
\newrobustcmd{\B}{\bfseries}

\renewcommand{\arraystretch}{1.3}

\DeclareMathOperator*{\argmin}{\arg\!\min}

\def\BibTeX{{\rm B\kern-.05em{\sc i\kern-.025em b}\kern-.08em
    T\kern-.1667em\lower.7ex\hbox{E}\kern-.125emX}}
    

\begin{document}


\title{Article Template \\
% {\footnotesize \textsuperscript{*}Note: Sub-titles are not captured in Xplore and
% should not be used}
% \thanks{Identify applicable funding agency here. If none, delete this.}
}

% 1\textsuperscript{st}

\author{\IEEEauthorblockN{Hao Wen}
\IEEEauthorblockA{\textit{Dept. Computer Science \& Technology} \\
\textit{Tsinghua University}\\
Beijing, China \\
wenh-06-10@tsinghua.edu.cn}
% \and
% \IEEEauthorblockN{Wenjian Yu}
% \IEEEauthorblockA{\textit{Dept. Computer Science \& Technology} \\
% \textit{Tsinghua University}\\
% Beijing, China \\
% yu-wj@tsinghua.edu.cn}
% \and
% \IEEEauthorblockN{Yuanqing Wu}
% \IEEEauthorblockA{\textit{JingDong Health Inc.}\\
% Beijing, China \\
% wuyuanqing@jd.com}
% \and
% \IEEEauthorblockN{Lu Wang}
% \IEEEauthorblockA{\textit{}\\
% Beijing, China \\
% @jd.com}
% \and
% \IEEEauthorblockN{Jun Zhao}
% \IEEEauthorblockA{\textit{JingDong Health Inc.}\\
% Beijing, China \\
% zhaojun10@jd.com}
% \and
% \IEEEauthorblockN{Xiaolong Liu}
% \IEEEauthorblockA{\textit{JingDong Health Inc.}\\
% Beijing, China \\
% liuxiaolong10@jd.com}
% \and
% \IEEEauthorblockN{Zhexiang Kuang}
% \IEEEauthorblockA{\textit{JingDong Health Inc.}\\
% Beijing, China \\
% kuangzhexiang@jd.com}
% \and
% \IEEEauthorblockN{Rong Fan}
% \IEEEauthorblockA{\textit{JingDong Health Inc.}\\
% Beijing, China \\
% fanrong18@jd.com}
% \and
% \IEEEauthorblockN{Shuai Yang}
% \IEEEauthorblockA{\textit{Jing Dong Health Inc.}\\
% Beijing, China \\
% yangshuai15@jd.com}
% \and
% \IEEEauthorblockN{Jeethan Jue Zhang}
% \IEEEauthorblockA{\textit{Jing Dong Health Inc.}\\
% Beijing, China \\
% zhangjue@jd.com}
% \and
% \IEEEauthorblockN{Lei Han}
% \IEEEauthorblockA{\textit{Jing Dong Health Inc.}\\
% Beijing, China \\
% hanlei70@jd.com}
}

\maketitle

% TODO:
% 1. use any of yolov4, efficientdet to perform new fine-tune experiment
% 2. try new activation functions (swish, mish) or new loss functions (diou, ciou) to train from scratch using COCO2017 and perform fine-tune using ACNE04

\begin{abstract}

\end{abstract}

\begin{IEEEkeywords}

\end{IEEEkeywords}

\section{Introduction}
\label{sec:intro}

\section{Related Work}
\label{sec:related_work}

\section{Method}
\label{sec:method}

\section{Experiments and Discussion}
\label{sec:experiments}

\section{Conclusions and Future Work}
\label{sec:conclusions_fw}


\section{Code availability}




\bibliographystyle{IEEEtran}
\bibliography{IEEEabrv, references}


% \begin{IEEEbiography}
% [{\includegraphics[width=1in, height=1.25in, clip, keepaspectratio]{bio_images/bio_wenhao.jpg}}]{Hao Wen}
% \end{IEEEbiography}


\end{document}


% TODO: citation for labelImg:
% Tzutalin. LabelImg. Git code (2015). https://github.com/tzutalin/labelImg
