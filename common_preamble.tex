\usepackage{cite}
\usepackage{academicons}
\usepackage{amsmath,amssymb,amsfonts}
\usepackage{algorithmic}
\usepackage{graphicx}
\usepackage{textcomp}
\usepackage{float}
\usepackage{xcolor}
\usepackage{tikz}
\usetikzlibrary{shapes,arrows,decorations.pathmorphing,backgrounds,positioning,fit,petri,calc,hobby}
\usepackage[colorlinks,linkcolor=black,anchorcolor=black,citecolor=black, urlcolor=blue]{hyperref}
\usepackage{caption}
\usepackage{subcaption}
\usepackage{cleveref}
\captionsetup[table]{skip = 3pt}
\captionsetup[subfigure]{subrefformat=simple}
\usepackage{multirow}
\usepackage{booktabs}
\usepackage{etoolbox,siunitx}
\usepackage{array}
\newcolumntype{P}[1]{>{\centering\arraybackslash}p{#1}}

\tikzset{%
  every neuron/.style={
    circle,
    draw,
    minimum size=0.5cm
  },
  neuron missing/.style={
    draw=none, 
    scale=2,
    text height=0.333cm,
    execute at begin node=$\vdots$
  },
}

\tikzstyle{block} = [rectangle, draw, fill = blue!20, text width = 8em, text centered, rounded corners, inner sep = 8pt, minimum height = 4em]
\tikzstyle{wideblock} = [rectangle, draw, fill = blue!20, text width = 13em, text centered, rounded corners, inner sep = 11pt, minimum height = 4em]
\tikzstyle{smallblock} = [rectangle, draw, fill = blue!20, text width = 5em, text centered, rounded corners, inner sep = 4pt, minimum height = 3em]
\tikzstyle{emptyblock} = [rectangle, text width = 8em, text centered, minimum height = 4em]

\newcommand*\samethanks[1][\value{footnote}]{\footnotemark[#1]}

\newcommand{\orcid}[1]{\href{https://orcid.org/#1}{\textcolor[HTML]{A6CE39}{\aiOrcid}}}
